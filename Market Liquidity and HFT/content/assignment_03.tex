\newpage
\section{Assignment 03}

In the standard principal-agent model, the agent's incentive to work hard depends on whether her compensation is fixed or variable. The income of the principal, in turn, is completely variable. Hence if the principal is also the agent, she is incentivized to work hard.
Surprisingly, \textcite{bandiera_18} find the opposite. They compare CEO working hours as a measure of effort between family CEOs that are majority owners of their firm (principals) and professional CEOs who do not hold majority ownership stakes (agents). Professional CEOs work 6 hours more per week compared to family CEOs.
Since the authors cannot explain the difference in hours worked by CEO and firm characteristics, they argue that there is a difference in preference for leisure between both CEO groups. 
The hypothesis is that family CEOs are wealthier and, thus, derive a smaller marginal utility from labor income than their counterparts. While \textcite{bandiera_18} provide supporting evidence for their argument, they cannot confirm their hypothesis at the CEO level. 
The explanation brought forward by \textcite{bandiera_18} is plausible but not necessarily exhaustive. In the following, I would like to argue for an alternative hypothesis. \\

An alternative explanation for the finding is the screening and selection process that CEOs go through. Hence, it addresses CEO characteristics.  
On the one side, professional CEOs must take early decisions of their career path, choosing the right education, doing multiple internships, and selecting an adequate full-time position, possibly involving multiple years of consulting experience. All those selection processes filter for certain managerial skills and most importantly dedication. Hence, professional CEOs tend to be highly passionate and ambitious individuals.
On the other side, the selection of family CEOs is typically less restrictive. While some successors follow the career path described above before taking over the family-owned firm, the opposite case is equally possible. Namely that the successor of a family firm does not desire and is unfit to take it over but has an obligation to do so in order to not disappoint the family heritage.

Empirically, the model specification of \textcite{bandiera_18} could account for the mechanism pointed out by adding more CEO characteristics. That is, whether the CEO visited a target university, worked in consulting, or the last salary before the CEO started working for the current firm. 
One potential problem of this model expansion is that more career-driven family CEOs might not respond to the survey entailing a selection bias.

Another potential test for my hypothesis is controlling for family firm governance. One governance example would be requiring successors to demonstrate their managerial abilities by building a career outside of the family firm before they are offered a position at the family firm. The hypothesis can be tested by comparing the hours worked in family firms with and without such governance put in place. 
More generally speaking, governance is a crucial component of a firm's success and is, thus, a missing variable in the analysis. \\

Finally, \textcite{bandiera_18} briefly mention public finance considerations regarding management-ownership practices. The mechanism presented above might be an argument against the stated policies reducing intra family transfer. Refining the considerations of the authors, nudging family firms to introduce governance requiring successors to have more external working experience, might be an interesting policy instrument.

In summary, \textcite{bandiera_18} propose wealth effects as the main driver for family CEOs working fewer hours than professional CEOs. I offer an alternative mechanism stressing the screening and selection process that both CEO types go through. 


